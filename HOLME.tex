\documentclass[conference]{IEEEtran}
\IEEEoverridecommandlockouts
% The preceding line is only needed to identify funding in the first footnote. If that is unneeded, please comment it out.
\usepackage{cite}
\usepackage{amsmath,amssymb,amsfonts}
\usepackage{algorithmic}
\usepackage{graphicx}
\usepackage{textcomp}
\usepackage{xcolor}
\def\BibTeX{{\rm B\kern-.05em{\sc i\kern-.025em b}\kern-.08em
    T\kern-.1667em\lower.7ex\hbox{E}\kern-.125emX}}
    
    
\begin{document}
    
\title{HOLME \\{\large
 An Application for Migrating Smart Home Configuration Using Matter
 }
}

\author{\IEEEauthorblockN{Kang Museong}
\IEEEauthorblockA{\textit{College of Engineering} \\
\textit{Hanyang University}\\
\textit{Dept.of Information Systems}\\
Seoul, Korea \\
bbibbi4808@hanyang.ac.kr}
\and
\IEEEauthorblockN{Kwon Hyuktae}
\IEEEauthorblockA{\textit{College of Engineering} \\
\textit{Hanyang University}\\
\textit{Dept.of Information Systems}\\
Seoul, Korea \\
kwon0111@hanyang.ac.kr}
\and
\IEEEauthorblockN{Lim Kyumin}
\IEEEauthorblockA{\textit{College of Engineering} \\
\textit{Hanyang University}\\
\textit{Dept.of Information Systems}\\
Seoul, Korea \\
mycheesepasta@gmail.com}
\and
\IEEEauthorblockN{Ha Seongwu}
\IEEEauthorblockA{\textit{College of Engineering} \\
\textit{Hanyang University}\\
\textit{Dept.of Information Systems}\\
Seoul, Korea \\
rockey6865@hanyang.ac.kr}
}

\maketitle

\begin{abstract}
Our team is strivingly working on developing a MATTER-based smart home configuration maintenance application called `HOLME.' The aim of this application is to make it convenient to bring the existing smart home configuration to locations other than one's own home.
Traditional smart home configuration have been constrained to specific applications under the umbrella of a single manufacturer. However, with the emergence of the MATTER protocol, it has become possible to integrate various IoT devices of different manufacturers seamlessly. Based on the MATTER protocol, we try to utilize the protocol to upload and download one's smart home settings to the server in any circumstances.
Our core objectives revolve around the following two points: (1) Convenience: We aim to enable users to utilize our app to easily import their well-configured and time-consuming smart home settings  to different places via QR code at once.
(2) Replaceability: When users go to a new place which do not have device(s) in their homes, we aim to align the environment to the pre-configured environment. We provide notifications and reports after the replacement process.
Through a B2B business model, our application allows users to conveniently migrate their smart home environments using QR codes, not only from one home to another but also to places like hotels.
Through this project, we aspire to expand the scope of traditional smart homes and develop a service that will play a pivotal role in the emerging sharing economy, particularly in shared housing scenarios in the near future.
\end{abstract}

\begin{IEEEkeywords}
HOLME, MATTER, Convenience, Replaceability, B2B, Sharing Economy
\end{IEEEkeywords}

\begin{table}[h]
\caption{Role Assignments}
\def\arraystretch{1.24} \small
    \begin{tabular}{|p{1.8cm}|p{1.4cm}|p{4.4cm}|}
        \hline
        Roles & Name & Task description and etc. \\ \hline
         User, \par Customer, \par Development \par manager & Kang \par Museong &User/Customer consider what features should be added from the perspective of users or customers. Development manager oversees the overall aspects of the project, such as project scheduling and planning, product and service quality. Additionally, they accurately grasp user requirements and manage and supervise the entire software engineering process, including software design, development, and testing. \\ \hline

	\end{tabular}
\end{table}


\begin{table}
\def\arraystretch{1.24} \small
    \begin{tabular}{|p{1.8cm}|p{1.4cm}|p{4.4cm}|}
        \hline
        AI \par Developer & Kwon \par Hyuktae  & AI developers create programs that adapt to the business's needs based on collected and analyzed data. In this service, they develop AI that makes recommendations based on users' past data. They design, develop, implement, and monitor AI systems and focus on data collection and data transformation architecture. \\ \hline
        
        Software \par Developer \par(Back-end) & Lim \par Kyumin & Backend developers consider the backend systems required for project development, utilizing databases and SQL queries. They manage the server-side and databases related to websites, web applications, or mobile solutions. As backend developers, they can work with various programming languages such as Python, Java, Node.js, and JavaScript. \\ \hline
        
       Software \par Developer\par(Front-end) & Ha \par Seongwu & Front-end developers design the UI/UX of applications to enhance the user experience, considering user convenience while implementing application designs. To accomplish this, proficiency in React-Native, TypeScript, and CSS is necessary.
 \\ \hline
    \end{tabular}
\end{table}


\section{INTRODUCTION}

\subsection{Motivation}
In 2003, when the global population was 6.3 billion, there were 500 million IoT devices. However, it is anticipated that by 2025, there will be a distribution of 1 trillion IoT devices among a global population of 8.1 billion. As the proliferation of IoT devices has rapidly progressed, ICT companies have individually developed their IoT platforms. However, an issue arose as each company created their own applications.

The existing smart home environment relied on devices from specific manufacturers and their proprietary applications. This resulted in users being tied to specific brands or apps when setting up their smart home environments. For example, a user with Samsung smart home devices who wanted to add LG smart home devices had to reset their existing Samsung smart home environment and download the LG smart home app, which was quite inconvenient for users.

MATTER was created with the aim of unifying these 'fragmented' IoT platforms, making it possible to control IoT devices from various manufacturers using a single protocol.

With the introduction of the new MATTER protocol, it has become possible to effectively integrate various IoT devices. Therefore, we intend to develop an application that leverages cloud servers and generative AI to enable users to easily retrieve the smart home environment they configured at home from different locations.

Our team's goal is to provide a feature that allows users to set up their initial smart home environment just once. This environment should remain easily maintainable when they travel, move, or go on business trips to different locations. This application is not only applicable to individual users but also to B2B scenarios, such as hotels and other accommodation providers. Furthermore, it is anticipated that it will play a pivotal role in shared economy models, like shared housing, in the future.

\subsection{Problem Statement}
\begin{itemize}
\item [1] In the modern world, where the penetration rate of IoT (Internet of Things) devices is steadily increasing, the need for services that manage these devices and operate the environment efficiently is becoming more important. \\
\item [2] As the percentage of international travel and business trips has increased since the end of COVID-19 and this has resulted in many people moving to other places, the need for services that can easily relocate and set up the environment of smart home devices is becoming more pronounced.\\
\item [3] When moving environments from an existing smart home, compatibility issues arise due to differences between the manufacturer's starting smart device and the target smart device, which makes it difficult for users to move their current convenient smart home settings to a new location. In this situation, there is a need for services and capabilities to easily transfer and set the environment beyond the manufacturer's constraints between the starting and target smart devices.
\\
\item [4] When you move away from the existing place where your smart home is configured and move to another place, the new configuration feels cumbersome. This increases the need for solutions that make it easier to move around and to facilitate the smooth transfer of smart home devices.
\\
\item [5] There are no applications to date that take into account the possibility of automatic replacement for devices that are not in a new location when you move your smart home environment to another location. This is where users experience inconvenience in relocating their environment, and there is a growing demand for new solutions and capabilities.
\\\item [6] Traditional voice notification services often tend to provide information in a way that lists hard and unnecessary content, making it difficult for users to effectively accept information and understand the situation. These limitations increase the need for improved user experience and more effective voice notification services.
\end{itemize}

\subsection{Target Customer}
\begin{itemize}
\item [1] Smart Home Owners\\
The main target audience for this project is individual smart home owners. They are people who want to effectively manage their smart home environment and migrate to other places, including smart home owners who are interested in moving their residences or bringing them up from other places.\\
\item [2] Hotels and properties\\
HOLME can also be delivered to hotels and properties through the B2B business model. This may include tourists and business travelers who want to experience a home-like smart environment at a hotel or property.\\
\item [3] IoT device maker\\
HOLME opens up the possibility of working with IoT device maker's products by providing solutions that integrate and maintain compatibility with various IoT devices. Also, companies looking for new business opportunities are potential target customers for HOLME.\\
\item [4] Sharing Economy Participants\\
Sharing economy participants will be very interested in solutions that can easily share smart home environments and move to other places. These solutions will provide shared and rental home owners and users with the opportunity to move freely and enjoy similar smart environments in new places while sharing the convenience of smart\\
\end{itemize}


\subsection{Research on Related Software}
\begin{itemize}
\item [1]ThinQ\\
ThinQ is LG Electronics' brand for smart devices and appliances, providing users with a more convenient smart home experience through features like smart control, AI integration, voice commands, home automation, and smart routines. This technology and brand are utilized in various LG products. Additionally, users can invite others to their registered spaces using QR codes and have the capability to register MATTER-supported devices. \\
\item [2]SmartThings\\
 SmartThings is a smart home automation and control platform developed by Samsung Electronics, allowing users to centrally control and connect smart devices and appliances. This platform offers features such as convenient remote control via a smartphone app, automation and routine settings, compatibility with various devices, interconnectivity, and integration with third-party systems, providing users with an integrated smart home experience.\\
\item [3]NUGU Smart Home\\
 NUGU Smart Home is SK Telecom's smart home platform that allows users to control household appliances and IoT devices through voice commands. Additionally, this platform provides apartment management services, including apartment news updates, filing complaints, access to shared entrances, and parking information services, among others.
\\
\item [4]GIGA Genie Home IoT\\
Giga Genie Smart Home is KT's smart home platform, enabling control of household appliances and IoT devices through voice commands. This platform collaborates with various companies to manage not only appliances like refrigerators but also devices such as boilers and cars.
\\
\item [5]Google Cloud IoT Core\\
Developed by Google, this platform provides a fully managed service and allows easy and secure connection, management, and data ingestion from globally dispersed devices.
\\
\item [6]AWS IoT Device Management\\
Provided by Amazon Web Services (AWS), this platform aims to facilitate the secure and efficient management of Internet of Things (IoT) devices. It offers tools and features to simplify the onboarding, organization, monitoring, and updating of IoT devices at scale.
\\
\item [7]Azure IoT Hub\\
Azure IoT Hub is a versatile and scalable cloud platform (IoT PaaS) that caters to multiple tenants. It comprises an IoT device registry, data storage, and robust security features. It also offers a service interface to facilitate IoT application development.
\\
\item [8]Nabu Casa\\
Nabu Casa is the company behind Home Assistant, a smart home automation platform that integrates and manages smart home devices and services. They offer cloud services for remote management and expansion of smart homes and provide a subscription-based service for storing IoT device settings and routines in the cloud, enabling remote management.
\\
\item [9]Hubitat\\
Hubitat is a home automation hub that supports Z-Wave and Zigbee protocols, providing local control and processing, customisability, multiple smart home device compatibility, and cost-effective features. In addition, Hubitat is a hub service that manages the smart home hub itself, allowing seamless routine and rule-setting between devices from different manufacturers. 
\\
\end{itemize}


\subsection{Expectation Effectiveness}
\begin{itemize}
\item [1] Expanding the range of smart homes\\
In addition to breaking away from traditional smart home environments (single manufacturer, tied to specific applications), HOLME extends the reach of smart home technology by expanding existing smart home environments to other places, not just at home. This allows users to easily bring up and manage their smart home environment outside of home.
\\
\item [2] Technological innovation and Competitive advantage\\
Based on the MATTER protocol, HOLME integrates various IoT devices into smart home applications and combines generative AI with the cloud. This provides an opportunity to lead technological innovation and gain a competitive edge in the existing smart home market.
\\
\item [3] Forward-looking\\
HOLME will be able to lead smart home technology to the wider market through cooperation with various accommodations such as hotels. In addition, it will play a key role in the shared housing platform in the upcoming trend of the shared economy.
\\
\end{itemize}

\subsection{Key Definitions}
\begin{itemize}
\item [1] Virtual space\\
1) Basic Assumption: Save the configuration settings and routines I'm currently using to `virtual space.’\\\\
2) If IoT devices set at home are set in a `virtual space’ called ``My House'', the setting of a `virtual space’ called ``My House'' can be imported into another space.
\\
\item [2] Logical hub\\
Backbone SW that gives arbitrary IoT devices the ability to act as a matter hub.
\\
\item [3] synchronization\\
It refers to the process of calling the settings of `virtual space' by scanning a QR code.
\\
\item [4] allocation\\
The function of connecting the IoT devices present in each room to their respective rooms within the hotel.
\\
\end{itemize}

\subsection{Scenario}
\begin{itemize}
\item [1)] A hotel entered into an agreement with HOLME and `allocated’ the existing IoT devices to their respective rooms.
\\
\item [2)] User (Customer) uses the HOLME app at home and villa to store and use smart home environments and routines in the form of each `virtual space.’
\\
\item [3)] The user has a business trip coming up, so he browsed a hotel reservation app and made a reservation at a hotel with the HOLME mark.
\\
\item [4)] When the user arrived at the hotel room, he scanned the QR code in the room, thereby `synchronization' that room with the user's `virtual space.’ During the 'sync' process, the user can choose which `virtual space' to `synchronization' with.
\\
\item [5)]Then, a new `virtual space’ that can manage the hotel room was copied into the HOLME app, and the existing settings were applied.
\\
\end{itemize}



\subsection{Profit Structure}
\begin{itemize}
\item [1] Certification and Hotel Partnership\\
HOLME can partner with the hotel to provide certification for the hotel's smart home environment. This certification is responsible for ensuring the quality and stability of the hotel's smart home service to the customer. The hotel may pay a fee to obtain and maintain this certification.
\\
\item [2] Marketing and Public relations agreements\\
Once the hotel is certified by HOLME, it can be used for marketing and promotion purposes.HOLME promotes the hotel in its own application, and the hotel can promote HOLME to mutual benefit.
\\
\item [3] Custom Solutions and Consulting\\
Providing customized smart home solutions for properties such as hotels, and providing consulting and integrated services for this purpose can generate revenue.
\\
\end{itemize}



\section{REQUIREMENT ANALYSIS}

\subsection{Common Features}
\begin{enumerate}


\item[1] Tutorial \\ 
The tutorial should be the first screens where HOLME is downloaded and shown. Users should be able to select the following options. 
\begin{itemize}
\item [1)] Skip the tutorial and sign up directly
\item [2)] View next page. If a user has reached the last page of the tutorial, the next step should be sign-up\\
\end{itemize}


\item[2] Sign-Up \\ 
HOLME needs four types of information to sign up for membership. These are phone numbers, passwords, name, and birth dates.
\begin{itemize}
\item [1)] Enter phone numbers\\
The phone number must be entered, and the phone number is verified through the carrier's authentication system to confirm whether the phone number is valid for membership registration. The phone number serves as an ID in the subsequent login process.
\item [2)] Enter passwords \\
Passwords must be entered and must be at least 8 characters long in combination of 3 or more of English uppercase/English lowercase/number/special characters. When the user enters the desired password, it is displayed in the form of ‘****’ on the screen, expressing information about it as [Unavailable/Safe/Dangerous].
\item [3)] Enter a name  \\
The name must be entered, and subsequently set to the default nickname at first login. The name is also used in ID search.
\item [4)] Enter birth dates \\
The birth dates must be entered, and a pop-up window is displayed every year to celebrate the birthday of the user. The date of birth is also used in ID search.\\
\end{itemize}


\item[3] Log in\\
There are two types of logins: 1) Local logins through HOLME membership, 2) SNS logins through SNS linkage.
\begin{itemize}
\item [1)] Local logins through HOLME membership
\begin{itemize}
\item [(1)] Local logins through HOLME membership
The system checks whether the ID and password entered by the user have filled the digits. If the number of digits is not filled, a warning message is shown in red.
\item [(2)] When the ID and password input by the user exist in the member database, the user succeeds in logging in. After that, it moves to the main page.
\item [(3)] If the phone number and password entered by the user do not exist in the member database, the user fails to log in and displays ``Non-existent member'' in the pop-up window.
\end{itemize}
\item [2)] SNS logins through SNS linkage 
\begin{itemize}
\item [(1)] Utilizes the Google, Apple, Facebook, Amazon, Naver, Kakao sign-up APIs.
\item [(2)] If the SNS login link is successful, the system must receive the user's name and date of birth. Then, go to the main page.\\
\end{itemize}
\end{itemize}


\item[4] Find ID\\
It is a function that exists for people who have lost their ID. HOLME's ID is based on the phone number, but you can add an email. Therefore, when you forget your phone number, you find your phone number using e-mail, and when you forget your e-mail, you find an e-mail based on the phone number. In case you don't remember both, you can find your ID by the name and date of birth registered at the time of membership registration.
\begin{itemize}
\item [1)]  The system receives an e-mail or telephone number. If the corresponding information exists in the user DB, a phone number or email is notified based on the corresponding information.
\item [2)] If the user does not know either e-mail or phone number, the system receives the name and date of birth and teaches the ID if the information exists in the user DB\\
\end{itemize}


\item[5] Resetting password
\begin{itemize}
\item [1)]  The system receives an ID to reset the password.
\item [2)] If the input ID does not exist in the user DB, a warning message will be displayed saying, ``Please enter your email or phone number correctly.''
\item [3)] When the input ID exists in the user DB, the system receives the user's name, date of birth, and phone number, and goes through the process of verifying whether the user is correct through authentication by the carrier.
\item [4)] When the user succeeds in identifying himself/herself, the system receives a password so that the user can reset the password. At this time, the password must have a length of at least 8 characters in a combination of at least three of the English uppercase/English lowercase/number/special characters. When the user enters the desired password, it is displayed in the form of ‘****’ on the screen, expressing information about it as [unavailable/safe/dangerous].\\
\end{itemize}

\item[6] Language change\\
This is a function that should be presented in the upper right corner of the login window, and the initial default value is Korean. You should be able to change this into another language.
\end{enumerate}



\subsection{User-Specific}
\begin{enumerate}

\item[1] Main page \\ 
The main page is responsible for `Virtual space'. The main page consists of Virtual space management, Virtual space setting, device addition, device operation, and routine execution.
\begin{itemize}
\item [1)] Virtual space management\\
The user should be able to add Virtual space, name, modify, and delete each Virtual space. In addition, the color should be set for each place so that the top color of the main page can be changed together.
\item [2)] Virtual space settings\\
Among the virtual spaces added by the user, the desired Virtual space is set as the `current place' and set as the main page. In addition, it should be possible to show a list of Virtual spaces created by users, and to change the `current place' into another Virtual space.
\item [3)] Device connection\\
Users should be able to call up all devices at once through QR code recognition.
\item [4)] Device manipulation\\
Users should be able to manipulate IoT devices located on the main page.
\item [5)] Run Routine\\
The user should be able to execute routines located on the main page.\\
\end{itemize}

\item[2] Menu bar\\
The menu bar is responsible for `my settings'. The menu bar consists of the following five types.
\begin{itemize}
\item [1)] Device Settings\\
Detailed settings can be stored in advance for each type of IoT device. Here, the devices are virtual devices, and detailed settings may be set even for devices that do not have them in reality. After connecting to the space, the actual devices are covered with pre-set settings and the device setting is performed with a connection.
\item [2)] Routine Settings

\begin{itemize}
\item [(1)]  Search Routine \\
The user should be able to search for pre-set routines.
\item [(2)] Add Routine\\
Users should be able to generate the desired routine by adding routines. Here, the routine means the operation of several actions. The user should be able to set the routine name and then, when this routine starts, set which actions should be executed. And when adding each action, it should be possible to add the desired action of the desired device through 'device and action search'.
\item [(2)] Edit Routine\\
The user should be able to edit the routine. This refers to changing the name of the routine, adding actions, deleting actions, and changing the order of actions.
\end{itemize}

\item [3)] Home\\
The user should be able to go to the main screen of the space where he is currently located by pressing the home button.
\item [4)] Report\\
Reports are generated when the user connects preset device settings and routine settings to a location. This is a specific report of what settings have been applied to automatically connected devices, which are replaceable and which are irreplaceable. It also generates a report when the routine is executed. This should inform the user that a device has performed a certain action in the course of executing the routine execution of the routine.
\item [5)] My HOLME\\
My HOLME plays the role of `set-up' in other applications
These include profile changes, nickname changes, notification settings, network connections, IoT service connections, application for accommodation manager, language changes, one-to-one inquiries, email ID changes, and password changes.\\
\end{itemize}

\item[3] Import settings\\
In conjunction with other IoT management applications, HOLME should be able to load a list or routine of devices previously used by other applications to HOLME.
\\

\item[4] Considering Replaceability\\
When a user loads their smart home environment, considering the likelihood that the configurations of all IoT devices may not be identical, interchangeability is taken into account.\\
(1) When function replication is possible: automatic connection and function execution.\\
(2) When function replication is not possible: submit a report.
\\

\end{enumerate}

\subsection{Hotel-Specific}
\begin{enumerate}
\item[1] Log in for Hotel Administrator\\ 
Hotel administrator register as members with ordinary members, but if they apply for hotel administrator authority through ``application for accommodation manager'', and certify it at HOLME, a new hotel management menu will be opened.\\
\item[2] Menu bar for Hotel Administrator\\
The menu bar for hotel managers consists of the following. Room management, device management by room, QR code management, inquiry
\begin{itemize}
\item [1)] Room management\\
The user should be able to add plave, name, modify, and delete each place. In addition, the color should be set for each place so that the top color of the main page can be changed together.
\item [2)] Device management by room\\
Hotel managers should be able to add, modify, and delete IoT devices that will be placed in rooms created through `room management'. This can be applied collectively to multiple rooms.
\item [3)] QR code management\\
For rooms where device management for each room has been completed, a QR code that can bring all devices connected to the room to HOLME at once should be generated. In addition, the administrator must be able to expire the QR code if desired.
\item [4)] Inquiry\\
Hotels should be able to proceed remotely by contacting HOLME for all processes, including room management, room-specific device management, and QR code management. It also serves as a consultation channel for hotel managers and HOLME\\
\end{itemize}
\end{enumerate}



\section{DEVELOPMENT ENVIRONMENT}

\subsection{Choice of Software Development Platform}
\begin{enumerate}
\item[1] Development Platform
\begin{itemize}
\item [1)] Windows\\
Windows provides a wide range of development tools and integrated development environments (IDEs) for creating various types of applications, including web applications, desktop applications, mobile apps, and games. This supports effective code editing, debugging, testing, deployment, and collaboration, ultimately enhancing developers' productivity.
Furthermore, Windows supports multiple programming languages and frameworks, allowing developers to choose their preferred languages and technologies to flexibly meet project requirements.
Windows offers a user-friendly and intuitive interface, making it easy for developers to configure and manage their development environments. A robust community and support system enable developers to share experiences and receive assistance.
Lastly, Windows continuously updates and improves, ensuring access to the latest technologies and tools, empowering developers to stay current and modernize their applications. Windows is recognized as a versatile platform suitable for various software development fields, playing a crucial role in turning developers' ideas into reality.
\item [2)] macOS\\
macOS is a highly regarded operating system in the field of software development, known for its user-friendly interface and exceptional versatility. This operating system offers several advantages to developers, and let's explore some of them. Firstly, macOS provides essential development tools and an integrated development environment (IDE) for creating a wide range of applications, including web applications, desktop applications, mobile apps, and games. Official IDEs like Xcode are powerful tools for application development across various platforms such as macOS, iOS, watchOS, and tvOS. They support tasks like code writing, debugging, testing, deployment, and collaboration, significantly enhancing developer productivity.
Additionally, macOS supports a variety of programming languages and frameworks, allowing developers to choose their preferred languages and technologies, making it flexible to adapt to project requirements. macOS offers an intuitive and user-friendly interface that simplifies development environment setup and project management. The active macOS developer community provides a platform for sharing experiences and collaboration among developers.
Finally, macOS ensures access to the latest technologies and tools through continuous updates and improvements. Apple's dedication to innovation provides developers with the necessary features to leverage the latest technologies and modernize their applications. For these reasons, macOS is recognized as an essential platform for software development, playing a significant role in turning ideas into reality.
\\
\end{itemize}

\item[2] Language / Framework

\begin{itemize}
\item [1)] Programmin Languages
\begin{itemize}
\item [(1)] Typescript\cite{Typescript}
\begin{figure}[h]
\centering
\includegraphics[width=.7\columnwidth]{img/DevEnv/TypeScript.png}
\centering
\caption{TypeScript} 
\end{figure}\\
TypeScript is a powerful tool developed by Microsoft, which is a superset of JavaScript. It provides static type checking, enhancing the development of robust and scalable applications. Introduced in 2012, TypeScript allows developers to detect errors at compile time, resulting in fewer bugs and improved code quality. Additionally, TypeScript offers advanced features such as interfaces, generics, and decorators, strengthening code organization and maintenance. It maintains full compatibility with JavaScript, enabling a seamless migration of existing JavaScript code to TypeScript, providing web developers with various options. TypeScript's advantages have led many enterprises to adopt it over JavaScript, regardless of project size, enhancing development productivity and code reliability. In summary, TypeScript is a valuable tool for modern web development, providing developers with better code quality and efficiency while simplifying project management.\\

\item [(2)] Kotlin\cite{Kotlin}
\begin{figure}[h]
\centering
\includegraphics[width=.7\columnwidth]{img/DevEnv/kotlin.png}
\centering
\caption{Kotlin} 
\end{figure}\\
Kotlin is a modern programming language developed by JetBrains and widely used as an alternative to Java. Kotlin provides developers with concise syntax and a stable type system, making it easier to write and maintain code efficiently. Moreover, it is extensively employed in Android app development and offers seamless interoperability with existing Java code.
The succinct syntax enhances project productivity and fosters collaboration by simplifying code writing and comprehension. The robust type system detects errors at compile time, boosting code reliability and reducing runtime errors. In the realm of Android app development, Kotlin enables more efficient application development and improved user experiences. Additionally, Kotlin seamlessly integrates with existing Java code, facilitating the modernization of legacy projects.
In summary, Kotlin, as a contemporary programming language, offers numerous advantages, providing developers with improved code quality and efficiency while simplifying project management.\\

\item [(3)] GO lang\cite{GO lang}
\begin{figure}[h]
\centering
\includegraphics[width=.7\columnwidth]{img/DevEnv/GoLang.png}\centering
\caption{GO lang} 
\end{figure}\\
Go language, developed by Google, is a programming language that offers a combination of simplicity and powerful features. It's a compiled language known for its fast execution speed and efficient memory management. Go's concise syntax makes it easy to write and maintain code, and it comes with a rich standard library that supports various tasks.
Furthermore, Go language emphasizes concurrency and supports parallel programming through lightweight threads known as goroutines. It provides a module system for simplified dependency management, enhancing project management and collaboration. Go is utilized in a wide range of fields, from server development to cloud computing, mobile apps, games, data analysis, and artificial intelligence.
Finally, Go language's fast compilation speed and small executable file sizes enable efficient development and deployment. As a result, many developers and companies choose Go to develop efficient and stable software.\\

\end{itemize}
\item [2)] Frameworks
\begin{itemize}
\item [(1)] React Native\cite{ReactNative}
\begin{figure}[h]
\centering
\includegraphics[width=.8\columnwidth]{img/DevEnv/ReactNative.png}
\caption{React Native} 
\end{figure}\\
React Native is a framework that utilizes JavaScript and the React library to develop mobile apps for both iOS and Android platforms. It offers the advantage of cross-platform app development while delivering performance and user experiences similar to native apps. React Native employs a component-based structure to build apps with modular components and supports hot reloading for quickly verifying code changes. Additionally, it integrates native modules for accessing hardware features and interacting with external services. Furthermore, it benefits from an active community and a wealth of open-source packages, providing extensive support and resources to developers. React Native stands as a powerful tool for efficiently developing mobile apps.\\

\item [(2)] Spring Boot\cite{SpringBoot}
\begin{figure}[h]
\centering
\includegraphics[width=\columnwidth]{img/DevEnv/SpringBoot.jpg}
\caption{Spring boot} 
\end{figure}\\
Spring Boot is a framework for easily developing Java-based web applications and microservices. This framework offers convenient configuration, an embedded web server, automatic setup, starter dependencies, monitoring and management capabilities, support for microservices, and access to a rich ecosystem of libraries and tools. Using Spring Boot, developers can rapidly build applications, reduce the complexity of configuration, and enhance productivity.\\

\item [(3)] Hibernate\cite{Hivernate}\\
\begin{figure}[h]
\centering
\includegraphics[width=.7\columnwidth]{img/DevEnv/hivernate.png}
\caption{HIBERNATE} 
\end{figure}
Hibernate is an open-source Object-Relational Mapping (ORM) framework for Java. It enables seamless interaction between Java objects and relational databases. Key highlights of Hibernate include database agnosticism, automatic table generation, an Object-Oriented Query Language (HQL), caching, built-in transaction management, and a strong community and ecosystem. In essence, Hibernate simplifies database operations in Java applications, offering flexibility, performance, and portability. \\

\item [(4)] gRPC\cite{gRPC}\\
\begin{figure}[h]
\centering
\includegraphics[width=.7\columnwidth]{img/DevEnv/Grpc.png}
\caption{gRPC} 
\end{figure}\\
gRPC is a high-performance Remote Procedure Call (RPC) framework developed by Google, designed to facilitate communication between services in various environments. gRPC uses Protocol Buffers for data exchange, offering an efficient binary format that allows message definitions to be shared across multiple programming languages. The framework supports multiple programming languages, making it possible for clients and servers written in different languages to communicate seamlessly. Operating based on the HTTP/2 protocol, gRPC provides efficient and fast communication, featuring features such as multiplexing, header compression, bidirectional communication, and more. Additionally, gRPC includes automatic code generation, simplifying developer tasks and ensuring type safety. Widely used in cloud and microservices architectures, gRPC supports efficient service-to-service communication.\\
\end{itemize}
\end{itemize}

\item[3] Cost Estimation\\
To implement HOLME, it was necessary to obtain data from the database or obtain real-time information from the server while communicating with the server in real-time. Therefore, real-time server hosting or multiple APIs were required. However, during the development process, we initially made efforts to utilize open APIs, free modules, and free servers.\\

\item[4] Development Environment
\begin{table}[h]
\def\arraystretch{1.24} \small
    \begin{tabular}{|p{1.2cm}|p{3.0cm}|p{3.6cm}|}
        \hline
        Name & Computer Resource & Version of OS, SW\\ \hline
         Kang \par Museong & Intel Core i5\par 16GB RAM memory & Windows 10 \par TexLive 2022 \\ \hline
        Kwon \par Hyuktae & Apple M1 Chip \par 16GB RAM memory & MacOS Ventura 13.5 \par Visual Studio Code 1.82.0 \par IntelliJ (LTS)\par spring boot (3.1.1) \\ \hline
        
         Lim \par Kyumin & Apple M2 Chip\par 16GB RAM memory & MacOS Ventura 13.4 \par GoLand (LTS) \par WebStorm (LTS)\par IntelliJ (LTS) \par React-Native (10.1.3) \par spring boot (3.1.1)\\ \hline
        
       Ha \par Seongwu & Intel Core i5 \par 8GB RAM memory & Windows 11 Home \par Visual Studio Code 1.82.0 \par React-Native (10.1.3)\par Android studio (LTS) \par Visual Studio Code 1.82.0 \\ \hline

	\end{tabular}
\end{table}

\item[5] Cloud Platform\\
We plan to use GCP (Google Cloud Platform) instead of AWS (Amazon Web Services) for a specific reason. We believe that GCP is more suitable for our needs because we are looking for instance types that are lightweight for hosting our servers, and in such cases, GCP is a better fit compared to AWS.
\end{enumerate}

\subsection{Software in use}
\begin{enumerate}
\item[1] visual Studio Code
\begin{figure}[h]
\centering
\includegraphics[width=.4\columnwidth]{img/DevEnv/VisualStudioCode.png}
\caption{Visual Studio Code} 
\end{figure}\\
Visual Studio Code (VS Code) is a highly popular integrated development environment (IDE) among developers. This convenient code editor is available for free and is known for its speed and lightweight nature, making it a preferred choice among users. VS Code supports various programming languages and offers excellent extensibility, allowing users to add the necessary features through extensions. Additionally, it provides intelligent code completion, debugging, Git integration, and a range of development tools to simplify coding tasks. With a user-friendly and intuitive interface, it offers an environment for programmers to work efficiently. For these reasons, VS Code stands as one of the most favored development tools among developers.\\

\item[2] IntelliJ
\begin{figure}[h]
\centering
\includegraphics[width=.4\columnwidth]{img/DevEnv/IntelliJ.jpg}
\caption{Visual Studio Code} 
\end{figure}\\
IntelliJ IDEA is a renowned integrated development environment (IDE) designed for Java developers. It is developed by JetBrains and is known for its robust features and user-friendly interface. IntelliJ IDEA offers a wide range of tools and functionalities to enhance Java application development. With intelligent code completion, comprehensive coding assistance, and advanced refactorings, developers can write high-quality code more efficiently. The IDE also supports a variety of programming languages and frameworks, making it a versatile choice for different projects. Furthermore, it provides excellent integration with popular version control systems and build tools, streamlining the development process. Overall, IntelliJ IDEA is a powerful and versatile IDE that caters to the needs of Java developers and beyond, making it a top choice in the development community.\\

\item[3] Android Studio 
\begin{figure}[h]
\centering
\includegraphics[width=.4\columnwidth]{img/DevEnv/AndroidStudio.jpg}
\caption{Android Studio} 
\end{figure}\\
Android Studio is an integrated development environment (IDE) developed by Google for Android application development. This IDE serves as a core tool for developing Android apps, providing a user-friendly interface and a rich ecosystem of plugins to make Android app development more accessible. Integrated with the Android SDK tools, Android Studio allows quick access to the latest Android APIs and features. It also offers an emulator for simulating and testing apps on various Android devices, along with robust debugging and performance profiling tools to support the development process. Android Studio provides tools and resources for app deployment and assists developers in building Android apps and publishing them on app stores like Google Play Store.\\

\item[4] WebStorm
\begin{figure}[h]
\centering
\includegraphics[width=.4\columnwidth]{img/DevEnv/WebStorm.jpg}
\caption{WebStorm} 
\end{figure}\\
WebStorm is a popular integrated development environment (IDE) designed specifically for web development. Developed by JetBrains, it offers a comprehensive set of tools for building modern web applications using web technologies such as HTML, CSS, and JavaScript. WebStorm provides a rich and intuitive coding environment with features like code completion, navigation, and refactoring, making web development more efficient and productive. It also offers built-in support for popular web frameworks and libraries, real-time code analysis, and debugging capabilities to help developers create high-quality web applications. With its extensive set of features and continuous updates, WebStorm is a go-to choice for web developers looking to streamline their workflow and build web applications with ease.\\

\item[5] GoLand
\begin{figure}[h]
\centering
\includegraphics[width=.4\columnwidth]{img/DevEnv/GoLand.jpg}
\caption{GoLand} 
\end{figure}\\
GoLand is an integrated development environment (IDE) developed by JetBrains, designed specifically for the Go programming language. It offers powerful tools for developers and programmers working with Go, enhancing their productivity and facilitating efficient code development.
This IDE provides various features and tools tailored to the Go language's specific characteristics. It includes robust code completion, refactoring, debugging, testing, and module support, making code writing easier and more efficient. Additionally, GoLand integrates project management and version control tools, simplifying complex tasks for developers.
GoLand also offers features such as static analysis, code inspections, and auto-completion to enhance code quality, supporting safe and efficient Go language development. It serves as a comprehensive tool for all developers and teams working with the Go language, aiding in managing and developing Go language projects effectively.\\

\item[6] PostgreSQL
\begin{figure}[h]
\centering
\includegraphics[width=.7\columnwidth]{img/DevEnv/PostgreSQL.jpeg}
\caption{PostgreSQL} 
\end{figure}\\
PostgreSQL, often known as Postgres, is a versatile and open-source relational database management system. Its adaptability stands out, allowing developers to customize data types and functions to meet specific project needs. PostgreSQL excels in data integrity and supports advanced concurrency control, ensuring data consistency in multi-user scenarios. With an active and supportive community, it receives regular updates and improvements, making it a reliable and high-performance choice for businesses and developers seeking an open-source RDBMS.\\

\item[7] LaTeX
\begin{figure}[h]
\centering
\includegraphics[width=.5\columnwidth]{img/DevEnv/LaTeX.png}
\caption{LaTeX} 
\end{figure}\\
LaTeX is a free typesetting system designed for creating professional documents, spanning various academic fields such as science, mathematics, and technology. It utilizes text files with commands to define the structure, formatting, tables, graphics, equations, references, and more in a document. Using these commands, you can compose your document, and a compiler is used to produce an output in the form of a PDF or other document formats.
In contrast to word processors, LaTeX offers professionalism and consistency in formatting, making it ideal for creating documents like research papers, academic theses, books, presentations, and more. LaTeX reduces the need to worry about layout, fonts, and paragraph divisions. It excels in typesetting mathematical equations, allowing you to beautifully represent complex mathematical notations.
Furthermore, LaTeX is well-known for its strong community support and various packages and styles available, enabling users to customize documents to meet specific requirements. Due to these features, it is widely used among researchers, students, writers, and engineers, facilitating the creation of professional, high-quality documents.\\

\item[8] GitHub
\begin{figure}[h]
\centering
\includegraphics[width=.5\columnwidth]{img/DevEnv/GitHub.png}
\caption{GitHub} 
\end{figure}\\
GitHub is a web-based platform and service for version control and collaboration. It is widely used by developers and teams to manage and track changes in their code, making it an essential tool for software development. GitHub provides a centralized platform where developers can store, manage, and collaborate on their source code, as well as track any modifications or issues related to their projects.
One of GitHub's core features is Git, a distributed version control system. Git enables developers to track changes, work on different aspects of a project simultaneously, and merge their work efficiently. GitHub adds a collaborative layer on top of Git, allowing multiple team members to work on a project collaboratively, making it easier to handle pull requests and code reviews.
GitHub hosts millions of public repositories, making it a valuable resource for open-source projects. It offers features like issues tracking, project management boards, and wikis, helping teams streamline their development processes. Moreover, GitHub Actions allows for automated workflows, further enhancing productivity.\\

\item[9] Notion
\begin{figure}[h]
\centering
\includegraphics[width=.5\columnwidth]{img/DevEnv/Notion.jpg}
\caption{Notion} 
\end{figure}\\
Notion is an all-in-one productivity platform for collaboration, note-taking, project management, and knowledge-based tasks. It caters to both personal note-taking and collaborative work, allowing users to scale up for larger projects and team-based tasks. Notion offers a wide range of features and flexibility to create a customized work environment tailored to the unique needs of its users.
Notion employs a block-based approach to organizing information, providing users with creative flexibility. Users can combine various block types, including text, images, videos, checklists, tables, calendars, databases, and more, to structure their content as needed. This goes beyond simple text documents, making it suitable for tasks such as project planning, task tracking, knowledge base creation, and more.
Furthermore, Notion excels in supporting team collaboration. Multiple users can simultaneously edit documents and share notes for real-time collaboration. It includes features like comments, to-do lists, calendar management, and other collaboration-related tools. With these diverse features and high levels of customization, Notion is widely adopted as an effective productivity tool for both personal and business use.\\

\item[10] Zoom
\begin{figure}[h]
\centering
\includegraphics[width=.5\columnwidth]{img/DevEnv/Zoom.jpg}
\caption{Zoom} 
\end{figure}\\
Zoom is an online video conferencing and collaboration software that is used for remote communication and collaboration. This platform offers various features such as video meetings, webinars, screen sharing, group chat, and file sharing, enabling users to work and communicate efficiently. It is widely used, especially to support remote work and education across regions and internationally.\\

\end{enumerate}





\begin{thebibliography}{9}
\bibitem{Typescript}  “Typescript,” https://www.typescriptlang.org/, 2023.
\bibitem{Kotlin}  “Kotlin,” https://kotlinlang.org/, 2023.
\bibitem{GO lang}  “GO lang,” https://go.dev/, 2023.
\bibitem{ReactNative}  “React Native,” https://reactnative.dev/, 2023.
\bibitem{SpringBoot}  “SpringBoot,” https://spring.io/projects/spring-boot, 2023.
\bibitem{Hivernate}  “Hivernate,” https://hibernate.org/, 2023.
\bibitem{gRPC}  “gRPC,” https://grpc.io/, 2023.
\end{thebibliography}

\end{document}